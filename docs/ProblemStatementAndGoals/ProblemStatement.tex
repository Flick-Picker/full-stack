\documentclass{article}

\usepackage{tabularx}
\usepackage{booktabs}

\title{Problem Statement and Goals\\\progname}

\author{\authname}

\date{}

%% Comments

\usepackage{color}

\newif\ifcomments\commentstrue %displays comments
%\newif\ifcomments\commentsfalse %so that comments do not display

\ifcomments
\newcommand{\authornote}[3]{\textcolor{#1}{[#3 ---#2]}}
\newcommand{\todo}[1]{\textcolor{red}{[TODO: #1]}}
\else
\newcommand{\authornote}[3]{}
\newcommand{\todo}[1]{}
\fi

\newcommand{\wss}[1]{\authornote{blue}{SS}{#1}} 
\newcommand{\plt}[1]{\authornote{magenta}{TPLT}{#1}} %For explanation of the template
\newcommand{\an}[1]{\authornote{cyan}{Author}{#1}}

%% Common Parts

\newcommand{\progname}{Flick Picker}
\newcommand{\authname}{Team 7, 7eam
\\ Talha Asif - asift
\\ Jarrod Colwell - colwellj
\\ Madhi Nagarajan - nagarajm
\\ Andrew Carvalino - carvalia    
\\ Ali Tabar - sahraeia
}     

\usepackage{hyperref}
    \hypersetup{colorlinks=true, linkcolor=blue, citecolor=blue, filecolor=blue,
                urlcolor=blue, unicode=false}
    \urlstyle{same}
                                


\begin{document}

\maketitle

\begin{table}[hp]
\caption{Revision History} \label{TblRevisionHistory}
\begin{tabularx}{\textwidth}{llX}
\toprule
\textbf{Date} & \textbf{Developer(s)} & \textbf{Change}\\
\midrule
September 25, 2022 & Jarrod Colwell & Initial Problem Statement Section\\
September 26, 2022 & Jarrod Colwell & Updated formatting and added environment\\
September 26, 2022 & Andrew Carvalino & Added Goals and Stretch Goals sections\\
... & ... & ...\\
\bottomrule
\end{tabularx}
\end{table}

\section{Problem Statement}

\subsection{Problem}

\hspace*{\parindent}7eam will be developing a system to enable individuals or groups to find movies, tv shows, or anime that are compatible with an individual or group based on a variety of factors.\newline


Current solutions exist, however, they are mostly catered towards the individual and have lacking support for groups. In order for a user to emulate a group, they would need to figure out what the common likes and dislikes of the group are, the time constraints of each individual, as well as any other important factors that can be inputted. This process is slow and will often cause certain members' preferences to be left out or devalued. 

\begin{itemize}
	\item Flick picker will allow for an individual to find a movie, tv show, or anime that fits their preferences.	
	\item Flick picker will allow for a group of individuals to find a movie, tv show, or anime that most closely fits the preferences of the group.
\end{itemize}



\subsection{Inputs}
\begin{itemize}
	\item Users will input various preferences regarding movies, tv shows, and anime. 
	\item Users will create a group that can consist of one or more people.
\end{itemize}

\subsection{Outputs}
\begin{itemize}
	\item Flick Picker will output a set of movies, tv shows, or anime that fits the combined preferences of the group as closely as possible.
\end{itemize}

\subsection{Stakeholders}

\begin{itemize}
	\item Dr. Spencer Smith
	\item 7eam
	\item Potential Users
\end{itemize}

\subsection{Environment}

Flick Picker will be available as a web application.

\section{Goals}

\subsection{Cater to Users' Interests}
Allow for a wide variety of different categories and subcategories that a user may choose, in order that the system may provide a list of relevant recommendations.

\subsection{Find Common Interests}
Find content that meets some common criteria of all users in a group.

\subsection{API Data Collection}
Will obtain all necessary information on shows and movies through API services.

\subsection{Group Compromise}
Should there be no overlapping interests between two or more users in a group, the system will either provide show or movie recommendations that are of a similar 
genre to what was selected (an ”inbetween” solution), use a tie-breaking mechanism (if one set of categories have more popularity in the group), or select options 
from the non-overlapping categories.

\subsection{Web App Compatibility}
Will work on multiple web browsers, such as Google Chrome, Firefox, etc. on desktop devices.

\subsection{Create Account and Login}
User will be able to create an account with their saved preferences and account settings that are stored and will persist for future logins.

\section{Stretch Goals}

\subsection{Mobile Web Usability}
System could be optimized for easier use on web browsers for mobile and tablet devices of various sizes.

\subsection{Mobile App}
An app may made available on the Apple and Android stores for download.

\subsection{Login with Other Accounts}
Ability to log on to the system with another account, such as Facebook or Google.

\subsection{Link with Streaming Accounts}
Allow the users to link their account to their Netflix, Disney+, etc. so the user can watch whatever shows or movies have been recommended.

\subsection{Show Ratings}
Displays the ratings of recommended content from different sources, such as IMDb.

\subsection{Separate Dev and Prod Deployment}
For the purpose of allowing developers to push up code without harming user experience if errors arise from it.

\end{document}