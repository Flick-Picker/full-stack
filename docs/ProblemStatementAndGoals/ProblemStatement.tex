\documentclass{article}

\usepackage{tabularx}
\usepackage{booktabs}

\title{Problem Statement and Goals\\\progname}

\author{\authname}

\date{}

%% Comments

\usepackage{color}

\newif\ifcomments\commentstrue %displays comments
%\newif\ifcomments\commentsfalse %so that comments do not display

\ifcomments
\newcommand{\authornote}[3]{\textcolor{#1}{[#3 ---#2]}}
\newcommand{\todo}[1]{\textcolor{red}{[TODO: #1]}}
\else
\newcommand{\authornote}[3]{}
\newcommand{\todo}[1]{}
\fi

\newcommand{\wss}[1]{\authornote{blue}{SS}{#1}} 
\newcommand{\plt}[1]{\authornote{magenta}{TPLT}{#1}} %For explanation of the template
\newcommand{\an}[1]{\authornote{cyan}{Author}{#1}}

%% Common Parts

\newcommand{\progname}{Flick Picker}
\newcommand{\authname}{Team 7, 7eam
\\ Talha Asif - asift
\\ Jarrod Colwell - colwellj
\\ Madhi Nagarajan - nagarajm
\\ Andrew Carvalino - carvalia    
\\ Ali Tabar - sahraeia
}     

\usepackage{hyperref}
    \hypersetup{colorlinks=true, linkcolor=blue, citecolor=blue, filecolor=blue,
                urlcolor=blue, unicode=false}
    \urlstyle{same}
                                


\begin{document}

\maketitle

\begin{table}[hp]
\caption{Revision History} \label{TblRevisionHistory}
\begin{tabularx}{\textwidth}{llX}
\toprule
\textbf{Date} & \textbf{Developer(s)} & \textbf{Change}\\
\midrule
Date1 & Name(s) & Description of changes\\
Date2 & Name(s) & Description of changes\\
... & ... & ...\\
\bottomrule
\end{tabularx}
\end{table}

\section{Problem Statement}

\wss{You should check your problem statement with the
\href{https://github.com/smiths/capTemplate/blob/main/docs/Checklists/ProbState-Checklist.pdf}
{problem statement checklist}.}
\wss{You can change the section headings, as long as you include the required information.}

\subsection{Problem}

\subsection{Inputs and Outputs}

\wss{Characterize the problem in terms of ``high level'' inputs and outputs.  
Use abstraction so that you can avoid details.}

\subsection{Stakeholders}

\subsection{Environment}

\wss{Hardware and software}

\section{Goals}

\subsection{Cater to Users' Interests}
Allow for a wide variety of different categories and subcategories that a user may choose, in order that the system may provide a list of relevant recommendations.

\subsection{Find Common Interests}
Find content that meets some common criteria of all users in a group.

\subsection{API Data Collection}
Will obtain all necessary information on shows and movies through API services.

\subsection{Group Compromise}
Should there be no overlapping interests between two or more users in a group, the system will either provide show or movie recommendations that are of a similar 
genre to what was selected (an ”inbetween” solution), use a tie-breaking mechanism (if one set of categories have more popularity in the group), or select options 
from the non-overlapping categories.

\subsection{Web App Compatibility}
Will work on multiple web browsers, such as Google Chrome, Firefox, etc. on desktop devices.

\section{Stretch Goals}

\subsection{Mobile Web Usability}
System could be optimized for easier use on web browsers for mobile devices.

\subsection{Mobile App}
An app may made available on the Apple and Android stores for download.

\end{document}