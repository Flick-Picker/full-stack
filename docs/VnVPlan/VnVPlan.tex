\documentclass[12pt, titlepage]{article}

\usepackage{booktabs}
\usepackage{tabularx}
\usepackage{hyperref}
\hypersetup{
    colorlinks,
    citecolor=blue,
    filecolor=black,
    linkcolor=red,
    urlcolor=blue
}
\usepackage[round]{natbib}

%% Comments

\usepackage{color}

\newif\ifcomments\commentstrue %displays comments
%\newif\ifcomments\commentsfalse %so that comments do not display

\ifcomments
\newcommand{\authornote}[3]{\textcolor{#1}{[#3 ---#2]}}
\newcommand{\todo}[1]{\textcolor{red}{[TODO: #1]}}
\else
\newcommand{\authornote}[3]{}
\newcommand{\todo}[1]{}
\fi

\newcommand{\wss}[1]{\authornote{blue}{SS}{#1}} 
\newcommand{\plt}[1]{\authornote{magenta}{TPLT}{#1}} %For explanation of the template
\newcommand{\an}[1]{\authornote{cyan}{Author}{#1}}

%% Common Parts

\newcommand{\progname}{Flick Picker}
\newcommand{\authname}{Team 7, 7eam
\\ Talha Asif - asift
\\ Jarrod Colwell - colwellj
\\ Madhi Nagarajan - nagarajm
\\ Andrew Carvalino - carvalia    
\\ Ali Tabar - sahraeia
}     

\usepackage{hyperref}
    \hypersetup{colorlinks=true, linkcolor=blue, citecolor=blue, filecolor=blue,
                urlcolor=blue, unicode=false}
    \urlstyle{same}
                                


\begin{document}

\title{Project Title: System Verification and Validation Plan for \progname{}} 
\author{Author Name}
\date{\today}
	
\maketitle

\pagenumbering{roman}

\section*{Revision History}
\begin{table}[hp]
	\caption{Revision History} \label{TblRevisionHistory}
	\begin{tabularx}{\textwidth}{llX}
		\toprule
		\textbf{Date} & \textbf{Developer(s)} & \textbf{Change}\\
		\midrule
		October 31 & Jarrod Colwell & Summary \& Objectives content added\\
		\bottomrule
	\end{tabularx}
\end{table}

\newpage

\tableofcontents

\listoftables
\wss{Remove this section if it isn't needed}

\listoffigures
\wss{Remove this section if it isn't needed}

\newpage

\section{Symbols, Abbreviations and Acronyms}

\renewcommand{\arraystretch}{1.2}
\begin{tabular}{l | l} 
  \toprule		
  \textbf{Symbol} & \textbf{Description}\\
  \midrule 
  T & Test\\
  V\&V & System Verification and Validation\\
  SRS & Software Requirements Specification\\
  UI & User Interface\\
  FR & Functional Requirement\\
  NFR & Non-Functional Requirement\\
  OAuth & Open Authorization\\
  \bottomrule
\end{tabular}\\

% \wss{symbols, abbreviations or acronyms -- you can simply reference the SRS \citep{SRS} tables, if appropriate}

\newpage

\pagenumbering{arabic}

This document contains the V\&{}V plan for Flick Picker, as well as the roadmap of when the tests will be implemented.

\section{General Information}

\subsection{Summary}
This document describes the plan to verify and validate that Flick Picker meets the defined requirements and specifications. Additionally, this document will validate that Flick Picker fulfills its intended purpose of recommending compatible Movies, TV Shows, or Anime to an individual or a group.

\subsubsection{Front End Testing}
\begin{itemize}
	\item Account Creation - Web page that facilitates user account creation
	\item User Preferences - Web page that facilitates user preference settings
	\item Group Creation - Web page that facilitates the creation of groups
	\item Recommendation - Web page that displays recommendations for an individual or group
\end{itemize}
\subsubsection{Back End Testing}
\begin{itemize}
	\item Database Access - Accessing the database to find user preferences or information pertaining to Movies, TV Shows, or Anime
	\item Recommendation Algorithm - The algorithm responsible for finding the best Movies, TV Shows, or Anime for an individual or group
	\item API Data - The accessing, storage, and usage of external data from various APIs
\end{itemize}

% \wss{Say what software is being tested.  Give its name and a brief overview of its general functions.}

\subsection{Objectives}
\subsubsection{Requirements}
The first objective involves verifying that Flick Picker meets requirements outlined in our SRS document and validating that the behaviour present is desirable. This includes the functional requirements (e.g. Authentication Requirements) and the non-functional requirements (e.g. Appearance Requirements). This objective will build confidence in the correctness of Flick Picker along with validating that security and usability standards are met.

\subsubsection{UI Elements}
The second objective of the V\&V document involves the validation of navigability and functionality of UI elements. Each menu described in the 'Front End Testing' section above must be functional for users. Additionally, users must be able to navigate to each page individually. This objective ensures that usability and response time standards are met. 

\subsubsection{External Connections}
The final objective of the V\&V document is to ensure that all external connections of Flick Picker (e.g. APIs, Database) work as intended. This objective ensures that the interoperability of Flick Picker is adequate. 

%\wss{State what is intended to be accomplished.  The objective will be around the qualities that are most important for your project.  You might have something like: ``build confidence in the software correctness,'' ``demonstrate adequate usability.'' etc.  You won't list all of the qualities, just those that are most important.}

\subsection{Relevant Documentation}

% \wss{Reference relevant documentation.  This will definitely include your SRS and your other project documents (MG, MIS, etc). You can include these even before they are written, since by the time the project is done, they will be written.}
\subsubsection{SRS}
\citet{SRS}

\subsubsection{MG}
\citet{MG}

\subsubsection{MIS}
\citet{MIS}

\section{Plan}

\wss{Introduce this section.   You can provide a roadmap of the sections to
  come.}

\subsection{Verification and Validation Team}

\wss{You, your classmates and the course instructor.  Maybe your supervisor.
  You shoud do more than list names.  You should say what each person's role is
  for the project.  A table is a good way to summarize this information.}

\subsection{SRS Verification Plan}

\wss{List any approaches you intend to use for SRS verification.  This may just
  be ad hoc feedback from reviewers, like your classmates, or you may have
  something more rigorous/systematic in mind..}

\wss{Remember you have an SRS checklist}

\subsection{Design Verification Plan}

\wss{Plans for design verification}

\wss{The review will include reviews by your classmates}

\wss{Remember you have MG and MIS checklists}

\subsection{Implementation Verification Plan}

\wss{You should at least point to the tests listed in this document and the unit
  testing plan.}

\wss{In this section you would also give any details of any plans for static verification of
  the implementation.  Potential techniques include code walkthroughs, code
  inspection, static analyzers, etc.}

\subsection{Automated Testing and Verification Tools}

\wss{What tools are you using for automated testing.  Likely a unit testing
  framework and maybe a profiling tool, like ValGrind.  Other possible tools
  include a static analyzer, make, continuous integration tools, test coverage
  tools, etc.  Explain your plans for summarizing code coverage metrics.
  Linters are another important class of tools.  For the programming language
  you select, you should look at the available linters.  There may also be tools
  that verify that coding standards have been respected, like flake9 for
  Python.}

\wss{The details of this section will likely evolve as you get closer to the
  implementation.}

\subsection{Software Validation Plan}

\wss{If there is any external data that can be used for validation, you should
  point to it here.  If there are no plans for validation, you should state that
  here.}

\section{System Test Description}
The tests outlined here will follow the following format:
\paragraph*{Test \#{}: Test Title}
\begin{itemize}
	\item[Control:] FR: Manual or Automated | NFR: Functional, Dynamic, Manual, or Static
	\item[Initial State:] Initial State
	\item[Input:] Input
	\item[Output:] Output
	\item[Derivation:] Test Case Derivation
	\item[Execution:] Automated Selenium Test | API Test
\end{itemize}
The execution of the test will be through automated selenium tests for the front-end and an API test library for the back-end.
	
\subsection{Tests for Functional Requirements}
The tests below are based directly on the FR sections from the SRS. Since these are mandatory requirements, they will all be automated, and the tests must pass on any new code change to prevent these requirements from being violated.

\subsubsection{Area of Testing: Authentication}
Contains all the tests for the Authentication FR.

\paragraph*{UI Auth T 1: User Login - Success}
\begin{itemize}
	\item[Control:] Automated
	\item[Initial State:] Empty email and password box, on the login screen. Pre-existing test customer exists. 
	\item[Input:] Email and password is filled correctly
	\item[Output:] Customer is logged in
	\item[Derivation:] Need a successful UI test on what a customer will see once they are logged in correctly
	\item[Execution:] Selenium Automated Test
\end{itemize}

\paragraph*{API Auth T 1: User Login - Success}
\begin{itemize}
	\item[Control:] Automated
	\item[Initial State:] Pre-existing test customer exists. 
	\item[Input:] `login' API is targeted with correct credentials filled in
	\item[Output:] Success response is returned (200)
	\item[Derivation:] Need a successful API test on how the login API will respond to correct information
	\item[Execution:] API Test
\end{itemize}

\paragraph*{UI Auth T 2: User Login - Failure}
\begin{itemize}
	\item[Control:] Automated
	\item[Initial State:] Empty email and password box, on the login screen. Pre-existing test customer exists. 
	\item[Input:] Email and password is filled incorrectly
	\item[Output:] Customer failed to login
	\item[Derivation:] Need a successful failure UI test on what a customer will see once they input their credentials incorrectly
	\item[Execution:] Selenium Automated Test
\end{itemize}

\paragraph*{API Auth T 2: User Login - Failure}
\begin{itemize}
	\item[Control:] Automated
	\item[Initial State:] Pre-existing test customer exists. 
	\item[Input:] `login' API is targeted with incorrect credentials filled in
	\item[Output:] Failure response is returned (401)
	\item[Derivation:] Need a successful API test on how the login API will respond to incorrect information
	\item[Execution:] API Test
\end{itemize}

\paragraph*{UI Auth T 3: User Sign-Up - Success}
\begin{itemize}
	\item[Control:] Automated
	\item[Initial State:] Empty email and password box, on the login screen.
	\item[Input:] Sign-up button is pressed, all boxes for sign-up are filled in correctly, confirm button is pressed
	\item[Output:] Customer account is created
	\item[Derivation:] Need a successful UI test on what a customer will see once they Sign-Up correctly
	\item[Execution:] Selenium Automated Test
\end{itemize}

\paragraph*{API Auth T 3: User Sign-Up - Success}
\begin{itemize}
	\item[Control:] Automated
	\item[Initial State:] Not Applicable
	\item[Input:] `signup' API is targeted with information correctly filled in
	\item[Output:] Success response is returned (200) and customer account is created
	\item[Derivation:] Need a successful API test on how the signup API will respond to correct information
	\item[Execution:] API Test
\end{itemize}

\paragraph*{UI Auth T 4: User Sign-Up - Failure}
\begin{itemize}
	\item[Control:] Automated
	\item[Initial State:] Empty email and password box, on the login screen.
	\item[Input:] Sign-up button is pressed, all boxes for sign-up are filled in incorrectly, confirm button is pressed
	\item[Output:] Areas where there is incorrect information is highlighted
	\item[Derivation:] Need a successful UI test on what a customer will see if they do not Sign-Up correctly
	\item[Execution:] Selenium Automated Test
\end{itemize}

\paragraph*{API Auth T 4: User Sign-Up - Failure}
\begin{itemize}
	\item[Control:] Automated
	\item[Initial State:] Not Applicable
	\item[Input:] `signup' API is targeted with information incorrectly filled in
	\item[Output:] Failure response is returned (422) and customer account is not created
	\item[Derivation:] Need a successful API test on how the signup API will respond to correct information
	\item[Execution:] API Test
\end{itemize}

\paragraph*{UI Auth T 5: User Logout}
\begin{itemize}
	\item[Control:] Automated
	\item[Initial State:] Test customer is logged in, on home page
	\item[Input:] Logout is pressed
	\item[Output:] Login screen displays
	\item[Derivation:] Need a UI test on what a customer will see if they log out
	\item[Execution:] Selenium Automated Test
\end{itemize}

\paragraph*{API Auth T 5: User Logout}
\begin{itemize}
	\item[Control:] Automated
	\item[Initial State:] Test customer is logged in
	\item[Input:] `logout' API is targeted with the test customer information
	\item[Output:] Success response is returned (500)
	\item[Derivation:] Need to test the logout API
	\item[Execution:] API Test
\end{itemize}

\paragraph*{UI Auth T 6: User OAuth Sign-Up/Login}
\begin{itemize}
	\item[Control:] Automated
	\item[Initial State:] Empty email and password box, on the login screen.
	\item[Input:] OAuth button is pressed
	\item[Output:] Correctly redirect to OAuth login
	\item[Derivation:] Need a successful UI test on what a customer will see if they choose to sign-up/login though an OAuth provider
	\item[Execution:] Selenium Automated Test
\end{itemize}

\subsubsection{Area of Testing: Profile/Group}
Contains all the tests for the Profile/Group FRs.

\paragraph*{UI PG T 1: Modify User Profile}
\begin{itemize}
	\item[Control:] Automated
	\item[Initial State:] On home page
	\item[Input:] Profile button is clicked, display name and password are updated, confirm is clicked
	\item[Output:] Confirmation that information is updated
	\item[Derivation:] Need a successful UI test on how a customer can update their profile
	\item[Execution:] Selenium Automated Test
\end{itemize}

\paragraph*{API PG T 1: Modify User Profile}
\begin{itemize}
	\item[Control:] Automated
	\item[Initial State:] Not Applicable
	\item[Input:] `updateProfile' API is targeted with data to update
	\item[Output:] Success status is returned (200)
	\item[Derivation:] Need a successful API test on how a customer can update their profile
	\item[Execution:] API Test
\end{itemize}

\paragraph*{UI PG T 2: Modify User Preferences}
\begin{itemize}
	\item[Control:] Automated
	\item[Initial State:] On home page
	\item[Input:] Profile button is clicked, show preferences are updated, confirm is clicked
	\item[Output:] Confirmation that preferences are updated
	\item[Derivation:] Need a successful UI test on how a customer can update their preferences
	\item[Execution:] Selenium Automated Test
\end{itemize}

\paragraph*{API PG T 2: Modify User Preferences}
\begin{itemize}
	\item[Control:] Automated
	\item[Initial State:] Not Applicable
	\item[Input:] `updatePreferences' API is targeted with data to update
	\item[Output:] Success status is returned (200)
	\item[Derivation:] Need a successful API test on how a customer can update their preferences
	\item[Execution:] API Test
\end{itemize}

\paragraph*{UI PG T 3: Create Group}
\begin{itemize}
	\item[Control:] Automated
	\item[Initial State:] On home page
	\item[Input:] Create Group button is clicked, name is input, confirm is clicked
	\item[Output:] Group is created with the specified name
	\item[Derivation:] Need a successful UI test on how a customer can create a group
	\item[Execution:] Selenium Automated Test
\end{itemize}

\paragraph*{API PG T 3: Create Group}
\begin{itemize}
	\item[Control:] Automated
	\item[Initial State:] Not Applicable
	\item[Input:] `createGroup' API is targeted with name specified
	\item[Output:] Success status is returned (200) and group is created
	\item[Derivation:] Need a successful API test on how a customer can create a group
	\item[Execution:] API Test
\end{itemize}

\paragraph*{UI PG T 4: Join Group}
\begin{itemize}
	\item[Control:] Automated
	\item[Initial State:] On home page
	\item[Input:] Invite is accepted through notifications
	\item[Output:] User is added to the group
	\item[Derivation:] Need a successful UI test on how a customer can join a group
	\item[Execution:] Selenium Automated Test
\end{itemize}

\paragraph*{API PG T 4: Join Group}
\begin{itemize}
	\item[Control:] Automated
	\item[Initial State:] Not Applicable
	\item[Input:] `acceptInvite' API is targeted
	\item[Output:] Success response is returned (200)
	\item[Derivation:] Need a successful API test on how a customer can join a group
	\item[Execution:] API Test
\end{itemize}

\paragraph*{UI PG T 5: Invite to Group}
\begin{itemize}
	\item[Control:] Automated
	\item[Initial State:] On home page
	\item[Input:] Group is selected, invite user through their username or email, confirm is clicked
	\item[Output:] User invite is sent
	\item[Derivation:] Need a successful UI test on how a customer can send an invite
	\item[Execution:] Selenium Automated Test
\end{itemize}

\paragraph*{API PG T 5: Invite to Group}
\begin{itemize}
	\item[Control:] Automated
	\item[Initial State:] Not Applicable
	\item[Input:] `sendInvite' API is targeted
	\item[Output:] Success response is returned (200)
	\item[Derivation:] Need a successful API test on how a customer can send an invite
	\item[Execution:] API Test
\end{itemize}

\subsubsection{Area of Testing: Recommendation}
Contains all the tests for the Recommendation FRs.

\paragraph*{UI R T 1: Receive Recommendations}
\begin{itemize}
	\item[Control:] Automated
	\item[Initial State:] On home page
	\item[Input:] Not Applicable
	\item[Output:] Observe if the home page is updated with recommendations
	\item[Derivation:] Need a successful UI test where a customer gets rotating recommendations
	\item[Execution:] Selenium Automated Test
\end{itemize}

\paragraph*{API R T 1: Receive Recommendations}
\begin{itemize}
	\item[Control:] Automated
	\item[Initial State:] Not Applicable
	\item[Input:] `viewRecommendations' API is targeted
	\item[Output:] Success response is returned (200) along with the list of shows
	\item[Derivation:] Need a successful API test where a customer gets rotating recommendations
	\item[Execution:] API Test
\end{itemize}

\paragraph*{UI R T 2: Rate Recommendations}
\begin{itemize}
	\item[Control:] Automated
	\item[Initial State:] On home page
	\item[Input:] Hover a movie and input recommendation
	\item[Output:] Movie should be updated with the recommendation
	\item[Derivation:] Need a successful UI test where a customer can update recommendations
	\item[Execution:] Selenium Automated Test
\end{itemize}

\paragraph*{API R T 2: Rate Recommendations}
\begin{itemize}
	\item[Control:] Automated
	\item[Initial State:] Not Applicable
	\item[Input:] `rateRecommendations' API is targeted along with enum data type
	\item[Output:] Success response is returned (200) and the movie recommendation is stored
	\item[Derivation:] Need a successful API test where a customer gets updated recommendations
	\item[Execution:] API Test
\end{itemize}

\paragraph*{API R T 3: Recommendation Match}
\begin{itemize}
	\item[Control:] Automated
	\item[Initial State:] Not Applicable
	\item[Input:] `findSharedRecommendation' API is targeted along with 5 user preferences
	\item[Output:] Success response is returned (200) and a show is selected with shared preferences
	\item[Derivation:] Need a successful API test where a group gets a recommended show
	\item[Execution:] API Test
\end{itemize}

\paragraph*{API R T 4: Recommendation Mismatch}
\begin{itemize}
	\item[Control:] Automated
	\item[Initial State:] Not Applicable
	\item[Input:] `findSharedRecommendation' API is targeted along with 5 user preferences, with entirely different preferences
	\item[Output:] Success response is returned (200) and a show is selected with shared the highest number of shared preferences
	\item[Derivation:] Need a successful API test where a group gets a recommended show where none of their preferences matched
	\item[Execution:] API Test
\end{itemize}

\subsection{Tests for Nonfunctional Requirements}

\wss{The nonfunctional requirements for accuracy will likely just reference the
  appropriate functional tests from above.  The test cases should mention
  reporting the relative error for these tests.}

\wss{Tests related to usability could include conducting a usability test and
  survey.}

\subsubsection{Area of Testing1}
		
\paragraph{Title for Test}

\begin{enumerate}

\item{test-id1\\}

Type: 
					
Initial State: 
					
Input/Condition: 
					
Output/Result: 
					
How test will be performed: 
					
\item{test-id2\\}

Type: Functional, Dynamic, Manual, Static etc.
					
Initial State: 
					
Input: 
					
Output: 
					
How test will be performed: 

\end{enumerate}

\subsubsection{Area of Testing2}

...

\subsection{Traceability Between Test Cases and Requirements}

\wss{Provide a table that shows which test cases are supporting which
  requirements.}

\section{Unit Test Description}
This section will be filled in after the MIS is completed. 

%\wss{Reference your MIS and explain your overall philosophy for test case
%  selection.}  
%\wss{This section should not be filled in until after the MIS has
%  been completed.}
%
%\subsection{Unit Testing Scope}
%
%\wss{What modules are outside of the scope.  If there are modules that are
%  developed by someone else, then you would say here if you aren't planning on
%  verifying them.  There may also be modules that are part of your software, but
%  have a lower priority for verification than others.  If this is the case,
%  explain your rationale for the ranking of module importance.}
%
%\subsection{Tests for Functional Requirements}
%
%\wss{Most of the verification will be through automated unit testing.  If
%  appropriate specific modules can be verified by a non-testing based
%  technique.  That can also be documented in this section.}
%
%\subsubsection{Module 1}
%
%\wss{Include a blurb here to explain why the subsections below cover the module.
%  References to the MIS would be good.  You will want tests from a black box
%  perspective and from a white box perspective.  Explain to the reader how the
%  tests were selected.}
%
%\begin{enumerate}
%
%\item{test-id1\\}
%
%Type: \wss{Functional, Dynamic, Manual, Automatic, Static etc. Most will
%  be automatic}
%					
%Initial State: 
%					
%Input: 
%					
%Output: \wss{The expected result for the given inputs}
%
%Test Case Derivation: \wss{Justify the expected value given in the Output field}
%
%How test will be performed: 
%					
%\item{test-id2\\}
%
%Type: \wss{Functional, Dynamic, Manual, Automatic, Static etc. Most will
%  be automatic}
%					
%Initial State: 
%					
%Input: 
%					
%Output: \wss{The expected result for the given inputs}
%
%Test Case Derivation: \wss{Justify the expected value given in the Output field}
%
%How test will be performed: 
%
%\item{...\\}
%    
%\end{enumerate}
%
%\subsubsection{Module 2}
%
%...
%
%\subsection{Tests for Nonfunctional Requirements}
%
%\wss{If there is a module that needs to be independently assessed for
%  performance, those test cases can go here.  In some projects, planning for
%  nonfunctional tests of units will not be that relevant.}
%
%\wss{These tests may involve collecting performance data from previously
%  mentioned functional tests.}
%
%\subsubsection{Module ?}
%		
%\begin{enumerate}
%
%\item{test-id1\\}
%
%Type: \wss{Functional, Dynamic, Manual, Automatic, Static etc. Most will
%  be automatic}
%					
%Initial State: 
%					
%Input/Condition: 
%					
%Output/Result: 
%					
%How test will be performed: 
%					
%\item{test-id2\\}
%
%Type: Functional, Dynamic, Manual, Static etc.
%					
%Initial State: 
%					
%Input: 
%					
%Output: 
%					
%How test will be performed: 
%
%\end{enumerate}
%
%\subsubsection{Module ?}
%
%...
%
%\subsection{Traceability Between Test Cases and Modules}
%
%\wss{Provide evidence that all of the modules have been considered.}
				
\bibliographystyle{plainnat}

\bibliography{../../refs/References}

\newpage

\section{Appendix}

This is where you can place additional information.

\subsection{Symbolic Parameters}

The definition of the test cases will call for SYMBOLIC\_CONSTANTS.
Their values are defined in this section for easy maintenance.

\subsection{Usability Survey Questions?}

\wss{This is a section that would be appropriate for some projects.}

\end{document}
