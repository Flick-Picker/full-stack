\documentclass[12pt, titlepage]{article}

\usepackage{amsmath, mathtools}

\usepackage[round]{natbib}
\usepackage{amsfonts}
\usepackage{amssymb}
\usepackage{graphicx}
\usepackage{colortbl}
\usepackage{xr}
\usepackage{hyperref}
\usepackage{longtable}
\usepackage{xfrac}
\usepackage{tabularx}
\usepackage{float}
\usepackage{booktabs}
\usepackage{multirow}
\usepackage[section]{placeins}
\usepackage{caption}
\usepackage{fullpage}

\hypersetup{
bookmarks=true,     % show bookmarks bar?
colorlinks=true,       % false: boxed links; true: colored links
linkcolor=red,          % color of internal links (change box color with linkbordercolor)
citecolor=blue,      % color of links to bibliography
filecolor=magenta,  % color of file links
urlcolor=cyan          % color of external links
}

\usepackage{array}

\externaldocument{../../SRS/SRS}

%% Comments

\usepackage{color}

\newif\ifcomments\commentstrue %displays comments
%\newif\ifcomments\commentsfalse %so that comments do not display

\ifcomments
\newcommand{\authornote}[3]{\textcolor{#1}{[#3 ---#2]}}
\newcommand{\todo}[1]{\textcolor{red}{[TODO: #1]}}
\else
\newcommand{\authornote}[3]{}
\newcommand{\todo}[1]{}
\fi

\newcommand{\wss}[1]{\authornote{blue}{SS}{#1}} 
\newcommand{\plt}[1]{\authornote{magenta}{TPLT}{#1}} %For explanation of the template
\newcommand{\an}[1]{\authornote{cyan}{Author}{#1}}

%% Common Parts

\newcommand{\progname}{Flick Picker}
\newcommand{\authname}{Team 7, 7eam
\\ Talha Asif - asift
\\ Jarrod Colwell - colwellj
\\ Madhi Nagarajan - nagarajm
\\ Andrew Carvalino - carvalia    
\\ Ali Tabar - sahraeia
}     

\usepackage{hyperref}
    \hypersetup{colorlinks=true, linkcolor=blue, citecolor=blue, filecolor=blue,
                urlcolor=blue, unicode=false}
    \urlstyle{same}
                                


\begin{document}

\title{Module Interface Specification for \progname{}}

\author{\authname}

\date{\today}

\maketitle

\pagenumbering{roman}

\section{Revision History}

\begin{tabularx}{\textwidth}{p{3cm}p{2cm}X}
\toprule {\bf Date} & {\bf Version} & {\bf Notes}\\
\midrule
January 18 & 0.1 & Added title, Module Decomposition table from Module Guide\\
January 18 & 0.2 & Updated section 2, 3, 4\\
April 5 & 1.0 &  Edits to MIS of modules - Madhi \\
\bottomrule
\end{tabularx}

~\newpage

\section{Symbols, Abbreviations and Acronyms}

See SRS Documentation at \url{https://github.com/Flick-Picker/full-stack/blob/develop/docs/SRS/SRS.pdf}

\begin{center}
	\renewcommand{\arraystretch}{1.2}
	\noindent 
	\begin{tabular}{l l p{7.5cm}} 
		\toprule 
		\textbf{Term} & \textbf{Abbreviation}\\ 
		\midrule
		True & T\\
		False & F\\
		\bottomrule
	\end{tabular}
\end{center}





\newpage

\tableofcontents

\newpage

\pagenumbering{arabic}

\section{Introduction}

The following document details the Module Interface Specifications for Flick Picker. Flick Picker is a web application that lets people select their preferences for watchable media and find recommendations of things to watch, being able to take multiple sets of preferences into account to find new media. Users can find new things to watch by themselves using only their preferences, or join user groups that find media based on what everyone likes in the group.

Complementary documents include the System Requirement Specifications
and Module Guide.  The full documentation and implementation can be
found at \url{https://github.com/Flick-Picker/full-stack}.

\section{Notation}


The structure of the MIS for modules comes from \citet{HoffmanAndStrooper1995},
with the addition that template modules have been adapted from
\cite{GhezziEtAl2003}.  The mathematical notation comes from Chapter 3 of
\citet{HoffmanAndStrooper1995}.  For instance, the symbol := is used for a
multiple assignment statement and conditional rules follow the form $(c_1
\Rightarrow r_1 | c_2 \Rightarrow r_2 | ... | c_n \Rightarrow r_n )$.

The following table summarizes the primitive data types used by \progname. 

\begin{center}
\renewcommand{\arraystretch}{1.2}
\noindent 
\begin{tabular}{l l p{7.5cm}} 
\toprule 
\textbf{Data Type} & \textbf{Notation} & \textbf{Description}\\ 
\midrule
character & char & a single symbol or digit\\
integer & $\mathbb{Z}$ & a number without a fractional component in (-$\infty$, $\infty$) \\
natural number & $\mathbb{N}$ & a number without a fractional component in [1, $\infty$) \\
real & $\mathbb{R}$ & any number in (-$\infty$, $\infty$)\\
boolean & bool & either T or F \\
\bottomrule
\end{tabular}
\end{center}

\noindent
The specification of \progname \ uses some common derived data types: sequences and strings. Sequences are lists filled with elements of the same data type. Strings are sequences of characters. Flick Picker also uses enumerated types, which are data types that hold a static set of constant values. In addition, \progname \ uses functions, which are defined by the data types of their inputs and outputs. Local functions are
described by giving their type signature followed by their specification.\pagebreak

Flick Picker also uses its own custom data types, summarized in the following table.



\begin{center}
	\renewcommand{\arraystretch}{1.2}
	\noindent 
	\begin{tabular}{l l p{7.5cm}} 
		\toprule 
		\textbf{Data Type} & \textbf{Notation} & \textbf{Description}\\ 
		\midrule
		Preferences & Preferences & a class that stores multiple boolean and enumerated values to keep track of user preferences such as genre or runtime\\
		User & User & a class that stores user information such as an ID, name, and Preferences \\
		Group & Group & a class that stores information about a user group, including an ID, owner, and user list. \\
		Authentication & Authentication & a class that stores information about a user's login, connecting a user ID to the appropriate password\\		
		\bottomrule
	\end{tabular} 
\end{center}





\section{Module Decomposition}

The following table is taken directly from the Module Guide document for this project.


\begin{table}[h!]
	\centering
	\begin{tabular}{p{0.3\textwidth} p{0.6\textwidth}}
		\toprule
		\textbf{Level 1} & \textbf{Level 2}\\
		\midrule
		
		{Hardware-Hiding Module} & ~ \\
		\midrule
		
		\multirow{4}{0.3\textwidth}{Behaviour-Hiding Module} & {Native Login Module}\\
		& Friends Module\\
		& Groups Module\\
		& Profile Module\\
		\midrule
		
		\multirow{3}{0.3\textwidth}{Software Decision Module} & {Matching Algorithm Module}\\
		& OAuth Login Module\\
		& API Module\\
		\bottomrule
		
	\end{tabular}
	\caption{Module Hierarchy}
	\label{TblMH}
\end{table}

\newpage

% MIS Module 3

\section{MIS of M3} \label{Module}

\subsection{Module}
Native Login Module

\subsection{Uses}
Types: User and Authentication

\subsection{Syntax}

\subsubsection{Exported Constants}
N/A

\subsubsection{Exported Access Programs}

\begin{center}
\begin{tabular}{p{3cm} p{4cm} p{4cm} p{3cm}}
\hline
\textbf{Name} & \textbf{In} & \textbf{Out} & \textbf{Exceptions} \\
\hline
signup() & String email, String password & - & Password does not meet requirements \\
login() & String email, String password & - & Unregistered email, Incorrect password \\
signOut() & - & - & - \\
\hline
\end{tabular}
\end{center}

\subsection{Semantics}

\subsubsection{State Variables}
int userId, String username, String email, String password

\subsubsection{Environment Variables}
N/A

\subsubsection{Assumptions}
User has a profile through an OAuth service or through our service, and our service is able to handle invalid sign-in attempts.

\subsubsection{Access Routine Semantics}

\noindent getUser():
\begin{itemize}
\item transition: User.id, User.name, User.email, Authentication.id, Authentication.password := userId, username, email, userUid, password
\item output: N/A
\item exception: N/A
\end{itemize}

\noindent signOut():
\begin{itemize}
\item transition: Authentication := Null
\item output: N/A
\item exception: N/A
\end{itemize}

\subsubsection{Local Functions}
N/A

% MIS Module 4

\section{MIS of M4} \label{Module}

\subsection{Module}
Friends Module

\subsection{Uses}
Types: User

\subsection{Syntax}

\subsubsection{Exported Constants}
N/A

\subsubsection{Exported Access Programs}

\begin{center}
\begin{tabular}{p{3cm} p{4cm} p{4cm} p{3cm}}
\hline
\textbf{Name} & \textbf{In} & \textbf{Out} & \textbf{Exceptions} \\
\hline
findFriend() & String searchName & List\textlangle User\textrangle users & No user with that name \\
addFriend() & - & - & - \\
deleteFriend() & - & - & - \\
requestFriend() & - & - & - \\
\hline
\end{tabular}
\end{center}

\subsection{Semantics}

\subsubsection{State Variables}
int friendId, String searchName

\subsubsection{Environment Variables}
N/A

\subsubsection{Assumptions}
The selected friend's account won't be deleted during the process of adding them

\subsubsection{Access Routine Semantics}

\noindent findFriend(searchName):
\begin{itemize}
\item transition: N/A
\item output: List\textlangle User\textrangle $\rightarrow$ User.name == searchName
\item exception: No user with the entered name
\end{itemize}

\noindent addFriend(friend):
\begin{itemize}
\item transition: User.friends.append(friendId), 
\item output: N/A
\item exception: N/A
\end{itemize}

\noindent deleteFriend(friend):
\begin{itemize}
\item transition: User.friends.remove(friend)
\item output: N/A
\item exception: N/A
\end{itemize}

\noindent requestFriend(friend):
\begin{itemize}
	\item transition: User.friendRequests(friend), friend.friendRequests(User)
	\item output: N/A
	\item exception: N/A
\end{itemize}

\subsubsection{Local Functions}
N/A

% MIS Module 5

\section{MIS of M5} \label{Module}

\subsection{Module}
Groups Module

\subsection{Uses}
Types: User, Group

\subsection{Syntax}

\subsubsection{Exported Constants}
N/A

\subsubsection{Exported Access Programs}

\begin{center}
\begin{tabular}{p{3cm} p{4cm} p{4cm} p{3cm}}
\hline
\textbf{Name} & \textbf{In} & \textbf{Out} & \textbf{Exceptions} \\
\hline
createGroup() & List\textlangle User\textrangle selectedUsers, String groupName & Group newGroup & - \\
deleteGroup() & Group selectedGroup & - & - \\
joinGroup() & - & Group newGroup & - \\
leaveGroup() & - & - & - \\
inviteToGroup() & User name & - & - \\
\hline
\end{tabular}
\end{center}

\subsection{Semantics}

\subsubsection{State Variables}
Group newGroup, int groupId, List\textlangle int\textrangle groupIds, Group invitedGroup

\subsubsection{Environment Variables}
N/A

\subsubsection{Assumptions}
N/A

\subsubsection{Access Routine Semantics}

\noindent createGroup():
\begin{itemize}
\item transition: groupIds := selectedUsers.id
\item output: Group newGroup := groupId, User.id, groupIds
\item exception: N/A
\end{itemize}

\noindent deleteGroup():
\begin{itemize}
\item transition: selectedGroup := Null
\item output: N/A
\item exception: N/A
\end{itemize}

\noindent joinGroup():
\begin{itemize}
\item transition: N/A
\item output: newGroup
\item exception: N/A
\end{itemize}

\noindent leaveGroup():
\begin{itemize}
\item transition: deletes user id from old group, but changes no state variable in the module
\item output: N/A
\item exception: N/A
\end{itemize}

\noindent inviteToGroup():
\begin{itemize}
\item transition: N/A
\item output: sends group info (name, id, user list, etc.) to the selected user
\item exception: N/A
\end{itemize}

\subsubsection{Local Functions}
N/A

% MIS Module 6

\section{MIS of M6} \label{Module}

\subsection{Module}
Profile Module

\subsection{Uses}
Types: User, Authentication, Preferences \\
Modules: M4

\subsection{Syntax}

\subsubsection{Exported Constants}
N/A

\subsubsection{Exported Access Programs}

\begin{center}
\begin{tabular}{p{3cm} p{4cm} p{4cm} p{3cm}}
\hline
\textbf{Name} & \textbf{In} & \textbf{Out} & \textbf{Exceptions} \\
\hline
editName() & String newName & - & Invalid name (swearing, length, etc.) \\
editEmail() & String newEmail & - & Invalid email \\
editPassword() & String newPassword & - & User does not authenticate password change \\
editFriends() & - & User.friends & - \\
editPreferences() & Preferences newPreferences & newPreferences & - \\
\hline
\end{tabular}
\end{center}

\subsection{Semantics}

\subsubsection{State Variables}
User.name, User.email, Authentication.password, User.friends, User.preferences

\subsubsection{Environment Variables}
N/A

\subsubsection{Assumptions}
The user is the one making changes, and not some other party

\subsubsection{Access Routine Semantics}

\noindent editName():
\begin{itemize}
\item transition: User.name := newName
\item output: N/A
\item exception: N/A
\end{itemize}

\noindent editEmail():
\begin{itemize}
\item transition: User.email := newEmail
\item output: N/A
\item exception: N/A
\end{itemize}

\noindent editPassword():
\begin{itemize}
\item transition: Authentication.password := newPassword
\item output: N/A
\item exception: N/A
\end{itemize}

\noindent editFriends():
\begin{itemize}
\item transition: uses Friends Module
\item output: User.friends
\item exception: N/A
\end{itemize}

\noindent editPreferences():
\begin{itemize}
\item transition: User.preferences := newPreferences
\item output: newPreferences
\item exception: N/A
\end{itemize}

\subsubsection{Local Functions}
editPreferences() will rely on functions that display and allow the user to chose the values for different Preferences keys

% MIS Module 8

\section{MIS of M8} \label{Module}

\subsection{Module}
Matching Algorithm Module

\subsection{Uses}
Types: Preferences

\subsection{Syntax}

\subsubsection{Exported Constants}
N/A

\subsubsection{Exported Access Programs}

\begin{center}
\begin{tabular}{p{3cm} p{4cm} p{4cm} p{3cm}}
\hline
\textbf{Name} & \textbf{In} & \textbf{Out} & \textbf{Exceptions} \\
\hline
recommendGroup() & Preferences group.preferences & List\textlangle String\textrangle shows & no matching results \\
recommendUser() & Preferences user.preferences & List\textlangle String\textrangle shows & no matching results \\
\hline
\end{tabular}
\end{center}

\subsection{Semantics}

\subsubsection{State Variables}
N/A

\subsubsection{Environment Variables}
N/A

\subsubsection{Assumptions}
N/A

\subsubsection{Access Routine Semantics}

\noindent recommendGroup():
\begin{itemize}
\item transition: N/A
\item output: List\textlangle String\textrangle shows
\item exception: N/A
\end{itemize}

\noindent recommendUser():
\begin{itemize}
\item transition: N/A
\item output: List\textlangle String\textrangle shows
\item exception: N/A
\end{itemize}

\subsubsection{Local Functions}
N/A

% MIS Module 9

\section{MIS of M9} \label{Module}

\subsection{Module}
OAuth Login Module

\subsection{Uses}
Types: User and Authentication

\subsection{Syntax}

\subsubsection{Exported Constants}
N/A

\subsubsection{Exported Access Programs}

\begin{center}
\begin{tabular}{p{3cm} p{4cm} p{4cm} p{3cm}}
\hline
\textbf{Name} & \textbf{In} & \textbf{Out} & \textbf{Exceptions} \\
\hline
getProfile() & String email, password & - & - \\
signOut() & - & - & - \\
\hline
\end{tabular}
\end{center}

\subsection{Semantics}

\subsubsection{State Variables}
int profileId, String profileName, String profileEmail

\subsubsection{Environment Variables}
N/A

\subsubsection{Assumptions}
User has a profile with the OAuth service they use to sign-in, with the provider of that service being able to handle invalid sign-in attempts.

\subsubsection{Access Routine Semantics}

\noindent getProfile():
\begin{itemize}
\item transition: User.id, User.name, User.email := profileId, profileName, profileEmail
\item output: N/A
\item exception: N/A
\end{itemize}

\noindent signOut():
\begin{itemize}
\item transition: Authentication := Null
\item output: N/A
\item exception: N/A
\end{itemize}

\subsubsection{Local Functions}
N/A as they are implemented within the OAuth (Google, Meta, or Apple)

% MIS Module 10

\section{MIS of M10} \label{Module}

\subsection{Module}
Matching Algorithm Module

\subsection{Uses}
Types: Preferences
Modules: M8

\subsection{Syntax}

\subsubsection{Exported Constants}
N/A

\subsubsection{Exported Access Programs}

\begin{center}
\begin{tabular}{p{3cm} p{4cm} p{4cm} p{3cm}}
\hline
\textbf{Name} & \textbf{In} & \textbf{Out} & \textbf{Exceptions} \\
\hline
groupData() & M8.recommendGroup & List\textlangle Recommendation\textrangle & no matching results \\
userData() & M8.recommendUser & List\textlangle Recommendation\textrangle  & no matching results \\
bestMatch() & Preferences & Recommendation & no matching results \\
\hline
\end{tabular}
\end{center}

\subsection{Semantics}

\subsubsection{State Variables}
N/A

\subsubsection{Environment Variables}
N/A

\subsubsection{Assumptions}
N/A

\subsubsection{Access Routine Semantics}

\noindent groupData():
\begin{itemize}
\item transition: N/A
\item output: List\textlangle Recommendation\textrangle 
\item exception: N/A
\end{itemize}

\noindent userData():
\begin{itemize}
\item transition: N/A
\item output: List\textlangle Recommendation\textrangle  
\item exception: N/A
\end{itemize}

\noindent bestMatch():
\begin{itemize}
	\item transition: max Recommendations[i].voteRating for i in Recommendations.length 
	\item output: Recommendation
	\item exception: N/A
\end{itemize}

\subsubsection{Local Functions}
N/A

\newpage

\bibliographystyle {plainnat}
\bibliography {../../../refs/References}

\newpage

\section{Appendix} \label{Appendix}

\wss{Extra information if required}

\end{document}