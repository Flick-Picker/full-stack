\documentclass[12pt, titlepage]{article}

\usepackage{fullpage}
\usepackage[round]{natbib}
\usepackage{multirow}
\usepackage{booktabs}
\usepackage{tabularx}
\usepackage{graphicx}
\usepackage{float}
\usepackage{hyperref}
\hypersetup{
    colorlinks,
    citecolor=blue,
    filecolor=black,
    linkcolor=red,
    urlcolor=blue
}

%% Comments

\usepackage{color}

\newif\ifcomments\commentstrue %displays comments
%\newif\ifcomments\commentsfalse %so that comments do not display

\ifcomments
\newcommand{\authornote}[3]{\textcolor{#1}{[#3 ---#2]}}
\newcommand{\todo}[1]{\textcolor{red}{[TODO: #1]}}
\else
\newcommand{\authornote}[3]{}
\newcommand{\todo}[1]{}
\fi

\newcommand{\wss}[1]{\authornote{blue}{SS}{#1}} 
\newcommand{\plt}[1]{\authornote{magenta}{TPLT}{#1}} %For explanation of the template
\newcommand{\an}[1]{\authornote{cyan}{Author}{#1}}

%% Common Parts

\newcommand{\progname}{Flick Picker}
\newcommand{\authname}{Team 7, 7eam
\\ Talha Asif - asift
\\ Jarrod Colwell - colwellj
\\ Madhi Nagarajan - nagarajm
\\ Andrew Carvalino - carvalia    
\\ Ali Tabar - sahraeia
}     

\usepackage{hyperref}
    \hypersetup{colorlinks=true, linkcolor=blue, citecolor=blue, filecolor=blue,
                urlcolor=blue, unicode=false}
    \urlstyle{same}
                                


\newcounter{acnum}
\newcommand{\actheacnum}{AC\theacnum}
\newcommand{\acref}[1]{AC\ref{#1}}

\newcounter{ucnum}
\newcommand{\uctheucnum}{UC\theucnum}
\newcommand{\uref}[1]{UC\ref{#1}}

\newcounter{mnum}
\newcommand{\mthemnum}{M\themnum}
\newcommand{\mref}[1]{M\ref{#1}}

\begin{document}

\title{System Design for \progname{}} 
\author{\authname}
\date{\today}

\maketitle

\pagenumbering{roman}

\section{Revision History}

\begin{tabularx}{\textwidth}{p{3cm}p{2cm}X}
\toprule {\bf Date} & {\bf Version} & {\bf Notes}\\
\midrule
January 18 & 1.0 & Added content to section 6.4, some potential content to 6.1\\
\bottomrule
\end{tabularx}

\newpage

\section{Reference Material}

This section records information for easy reference.

\subsection{Abbreviations and Acronyms}

\renewcommand{\arraystretch}{1.2}
\begin{tabular}{l l} 
  \toprule		
  \textbf{symbol} & \textbf{description}\\
  \midrule 
  \progname & Explanation of program name\\
  \wss{...} & \wss{...}\\
  \bottomrule
\end{tabular}\\

\newpage

\tableofcontents

\newpage

\listoftables

\listoffigures

\newpage

\pagenumbering{arabic}

\section{Introduction}

\wss{Include references to your other documentation}

\section{Purpose}

\wss{Purpose of your design documentation}

\wss{Point to your other design documents}

\section{Scope}

\wss{Include a figure that show the System Context (showing the boundary between
your system and the environment around it.)}

\section{Project Overview}

\subsection{Normal Behaviour}
%\begin{itemize}
%	\item \textbf{Login Module:} Provided the email and password, this module will communicate with the login database to authorize or deny the user's login.
%	\item \textbf{Profile module:} Flick picker will allow for an individual to find a movie, tv show, or anime that fits their preferences.
%	\item Flick picker will allow for a group of individuals to find a movie, tv show, or anime that most closely fits the preferences of the group.
%\end{itemize}

\subsection{Undesired Event Handling}

\wss{How you will approach undesired events}

\subsection{Component Diagram}

\subsection{Connection Between Requirements and Design} \label{SecConnection}

\subsubsection{Connection Between Authentication Requirements and Design}
\hspace*{14pt} For email and password authentication, the user's email and a hashed form of their password are stored in the database. Upon login request, the inputted email and password will be compared to the pairs in the database and the login will either succeed or fail depending on whether a match is found. 

Google and Facebook OAuth will provide authorization and an email to Flick Picker which will have / create an account with this email. If an account with this email already exists using OAuth, the user is logged in successfully. If an account using the previous method (email and password) exists, the user will be prompted to login using their email and password as above. If an account with this email does not exist, the user will be logged in and a new OAuth account will be created. They will then be brought to the new user page (creation of username, preferences, etc.). 

Upon clicking the logout button, the user will be brought to the login screen and their token is removed.

\subsubsection{Connection Between Profile/Group Requirements and Design}


\subsubsection{connection Between Recommendation Requirements and Design}
\hspace*{14pt} Daily, Flick Picker will retrieve information about the top 3000 most popular Movies, TV Shows, and Anime and store that information for quick access. This information will be used in the vast majority of all recommendations. If a recommendation cannot be found within these Movies, TV Shows, or Anime, additional queries will be sent to the relevant APIs to find a recommendation.

During the recommendation process for a group, individual users reflect their desire to watch a recommendation using the 'like', 'neutral', or 'dislike' buttons. This information will be stored alongside their preferences to aid in future recommendations for both the group and the individuals. 

Given user permission, Flick Picker will send emails to all members in a group once a recommendation has been chosen. 

\section{User Interfaces}

\wss{Design of user interface for software and hardware.  Attach an appendix if
needed. Drawings, Sketches, Figma}

\section{Design of Communication Protocols}

\wss{If appropriate}

\section{Timeline}

\wss{Schedule of tasks and who is responsible}

% \bibliographystyle {plainnat}
% \bibliography{../../../refs/References}

\newpage{}

\appendix

\section{Interface}

\wss{Include additional information related to the appearance of, and
interaction with, the user interface}

\section{Mechanical Hardware}

\section{Electrical Components}

\section{Communication Protocols}

\section{Reflection}

The information in this section will be used to evaluate the team members on the
graduate attribute of Problem Analysis and Design.  Please answer the following questions:

\begin{enumerate}
  \item What are the limitations of your solution?  Put another way, given
  unlimited resources, what could you do to make the project better? (LO\_ProbSolutions)
  \item Give a brief overview of other design solutions you considered.  What
  are the benefits and tradeoffs of those other designs compared with the chosen
  design?  From all the potential options, why did you select documented design?
  (LO\_Explores)
\end{enumerate}

\subsection{Talha Reflection}
\begin{enumerate}
	\item Our solution's two most significant limitations are the number of developers and time constraints. As 7eam only has five developers with solely software engineering experience, we need to learn how to market a product. We also need multiple years of development experience to avoid pitfalls that individuals who have worked in the industry would know to avoid. If we were given an unlimited number of individuals to make an entire development team with dedicated teams per dev-op, frontend, backend, and business teams, the application would see a much more significant influx of users and have an incredibly refined design. Furthermore, we are constrained by time, not just by the due dates but by how much content we five individual developers can put in. We could refine the application incredibly well with infinite time and no other resources.
	
	\item Another design solution was a minor decision about pairing and preferences together. The benefit is that they go hand-in-hand, where users are directly linked to their preferences. The downside being it adds an overhead to updating the preferences list, and since we are going to take into account user feedback for the preferences, it could go through a handful of updates. Therefore splitting it off into a module in isolation and linking the user is a safer implementation.
\end{enumerate}

\end{document}