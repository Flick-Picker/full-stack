\documentclass[12pt]{article}

\usepackage{booktabs}
\usepackage{tabularx}

\title{Hazard Analysis\\\progname}

\author{\authname}
\date{\today}

%% Comments

\usepackage{color}

\newif\ifcomments\commentstrue %displays comments
%\newif\ifcomments\commentsfalse %so that comments do not display

\ifcomments
\newcommand{\authornote}[3]{\textcolor{#1}{[#3 ---#2]}}
\newcommand{\todo}[1]{\textcolor{red}{[TODO: #1]}}
\else
\newcommand{\authornote}[3]{}
\newcommand{\todo}[1]{}
\fi

\newcommand{\wss}[1]{\authornote{blue}{SS}{#1}} 
\newcommand{\plt}[1]{\authornote{magenta}{TPLT}{#1}} %For explanation of the template
\newcommand{\an}[1]{\authornote{cyan}{Author}{#1}}

%% Common Parts

\newcommand{\progname}{Flick Picker}
\newcommand{\authname}{Team 7, 7eam
\\ Talha Asif - asift
\\ Jarrod Colwell - colwellj
\\ Madhi Nagarajan - nagarajm
\\ Andrew Carvalino - carvalia    
\\ Ali Tabar - sahraeia
}     

\usepackage{hyperref}
    \hypersetup{colorlinks=true, linkcolor=blue, citecolor=blue, filecolor=blue,
                urlcolor=blue, unicode=false}
    \urlstyle{same}
                                


\begin{document}

\maketitle

~\newpage \pagenumbering{roman}

\tableofcontents

~\newpage

\section*{Revision History}
\begin{table}[hp]
	\caption{Revision History} \label{TblRevisionHistory}
	\begin{tabularx}{\textwidth}{llX}
		\toprule
		\textbf{Date} & \textbf{Developer(s)} & \textbf{Change}\\
		\midrule
		October 17 & Jarrod Colwell & Created document structure\\
		October 17 & Talha Asif & Modifying Doc Structure\\
		... & ... & ...\\
		\bottomrule
		\end{tabularx}
\end{table}

~\newpage \pagenumbering{arabic}

\section{Introduction}
Before going any further with system design, it is crucial to conduct a hazard analysis of the system from an engineering perspective. The goal is to identify critical safety concerns the application users could face and the solutions to them. Hazards will be determined using the Failure Modes and Effects Analysis (FMEA) for Flick Picker.

\section{Scope and Purpose}
a

\section{Background}
a

\section{System Boundary}
a

\section{Scope of Hazard Analysis}
a

\section{Definition of Hazard}
a

\section{Critical Assumptions}
a

\section{Failure Modes and Effects Analysis}
a

\section{Safety Requirements}
a

\section{Roadmap}
a

\end{document}