\documentclass[12pt]{article}

\usepackage{booktabs}
\usepackage{tabularx}
\usepackage{pdflscape}
\usepackage{arydshln}

\title{Hazard Analysis\\\progname}

\author{\authname}
\date{\today}

%% Comments

\usepackage{color}

\newif\ifcomments\commentstrue %displays comments
%\newif\ifcomments\commentsfalse %so that comments do not display

\ifcomments
\newcommand{\authornote}[3]{\textcolor{#1}{[#3 ---#2]}}
\newcommand{\todo}[1]{\textcolor{red}{[TODO: #1]}}
\else
\newcommand{\authornote}[3]{}
\newcommand{\todo}[1]{}
\fi

\newcommand{\wss}[1]{\authornote{blue}{SS}{#1}} 
\newcommand{\plt}[1]{\authornote{magenta}{TPLT}{#1}} %For explanation of the template
\newcommand{\an}[1]{\authornote{cyan}{Author}{#1}}

%% Common Parts

\newcommand{\progname}{Flick Picker}
\newcommand{\authname}{Team 7, 7eam
\\ Talha Asif - asift
\\ Jarrod Colwell - colwellj
\\ Madhi Nagarajan - nagarajm
\\ Andrew Carvalino - carvalia    
\\ Ali Tabar - sahraeia
}     

\usepackage{hyperref}
    \hypersetup{colorlinks=true, linkcolor=blue, citecolor=blue, filecolor=blue,
                urlcolor=blue, unicode=false}
    \urlstyle{same}
                                


\begin{document}

\maketitle

~\newpage \pagenumbering{roman}

\tableofcontents

~\newpage

\section*{Revision History}
\begin{table}[hp]
	\caption{Revision History} \label{TblRevisionHistory}
	\begin{tabularx}{\textwidth}{llX}
		\toprule
		\textbf{Date} & \textbf{Developer(s)} & \textbf{Change}\\
		\midrule
		October 17 & Jarrod Colwell & Created document structure\\
		October 17 & Talha Asif & Modifying Doc Structure\\
		October 17 & Talha Asif & Added introduction section content\\
		October 17 & Jarrod Colwell & Added scope and purpose section content\\
		October 19 & Andrew Carvalino & Definition of Hazard and Critical Assumptions\\
		October 19 & Talha Asif & Adding Section 8\\
		October 19 & Ali Tabar & Adding Sections 5 and 6\\
    	October 19 & Madhi Nagarajan & Adding Sections 3 and 4\\
		October 19 & Jarrod Colwell & Section 1-4 editing\\
		October 19 & Jarrod Colwell & Section 5 editing \\

		\bottomrule
		\end{tabularx}
\end{table}

~\newpage \pagenumbering{arabic}

\section{Introduction}
Before going any further with system design, it is crucial to conduct a hazard analysis of the system from an engineering perspective. The goal is to identify critical safety concerns the users of the application could face, and the solutions to them. Hazards will be identified and eliminated or mitigated using the Failure Modes and Effects Analysis (FMEA).

\section{Scope and Purpose}
This document covers the various areas in which the system is most vulnerable, including but not limited to:
\begin{itemize}
	\item External Resource Integration Points
	\item Server Communication
	\item TODO: Add more here or delete
\end{itemize}
Along with identifying the vulnerable areas of the system, this document also covers the strategies, both elimination and mitigation, and new security requirements to reduce or eliminate the impact that these hazards have.

\section{Background}
Flick Picker is a web application that finds the most compatible movie, TV show, or Anime for an individual user or a group of users. Users will have the ability to set their preferences related to TV Shows, Movies, or Anime. Based on these preferences, the system will produce personalized recommendations for the individual user or the group.

\section{System Boundary}
The list below identifies the various components of the system:
\begin{enumerate}
	\item Web Application
	\begin{enumerate}
	  \item Authentication: Verifies and logs the user into the system.
	  \item Profile Management: Stores and manages the user's profile, including their username, preferences, groups etc. Note that this data is stored
	  \item Recommendation System: Provides movie/TV show recommendations to users and groups.
	\end{enumerate}
	\item The user's Physical Device (Laptop or Phone)
	\item External APIs (OMDb, MyAnimeList etc.): Our application requires these APIs to collect movie and TV show records.
	\item Database: Storing user data on our database, through Firebase.
	\item Deployments: Builds and deployments will be managed by Jenkins/GitHub Workflow.
	
\end{enumerate}

\noindent The system boundary includes the entire Flick Picker Application, and application database. Note that user's device and APIs are external elements, therefore not part of the system boundary. Firebase/Google maintains the uptime of our application and database. We also make use of Jenkins/GitHub Workflow for CI/CD of our application.

\section{Scope of Hazard Analysis}
This document will identify safety concerns and solutions that users may face via:
\begin{itemize}
	\item Defining what a hazard is in this context
	\item Stating the critical assumptions that are being made by the system
	\item Providing a Failure Modes and Effects Analysis of the components of the system
	\item Outlining the safety requirements that are a byproduct of that analysis
	\item Outlining a roadmap of when the hazard analysis may be consulted or further adjusted
\end{itemize} 

\section{Definition of Hazard}
A hazard, as defined by Nancy Leveson, is a property or condition in the system, that may cause some sort of loss when combined with an environmental condition.

\section{Critical Assumptions}
\begin{enumerate}
	\item System will not have direct access to users' hardware (ex. specific CPU registers)
	\item Files will not be downloaded onto the users' device without the explicit consent of the user (should that be a feature of the system)
	\item Users' private information will not be sold or intentionally disclosed to any third parties
\end{enumerate}

\begin{landscape}
\newpage
\section{Failure Modes and Effects Analysis}
Below are tables containing the full Failure Modes and Effects Analysis.
\label{table1}
\begin{table}[hp]
	\caption{Failure Modes and Effects 1} \label{TblFMEA1}
	\begin{tabularx}{\linewidth}{| p{0.185\textwidth} | X | X | p{0.25\textwidth} | X | l |}
		\hline
		Component & Failure Modes & Effects of Failure & Causes of Failure & Recommended Actions & SR \\
		\hline
		Database & Data is deleted on accident & All user data is lost & Database Failure & Regular backups exist where data can be rolled back on demand & IR1, IR2, IR3 \\
		\hdashline
		~ & Data is unavailable & User cannot access data & Database Failure & Refer Above & IR7 \\
		\hdashline
		Profile Management & Data is modified incorrectly & User data is not updated & Database Failure & System alerts if data is not modified when requested & IR2 \\
		\hline
		Authentication & User cannot login & User cannot view recommendations or friends & Authentication Failure & Use the correct credentials & AR1, PR1 \\
		\hline
		\end{tabularx}
\end{table}
\newpage
\label{table2}
\begin{table}[hp]
	\caption{Failure Modes and Effects 2} \label{TblFMEA2}
	\begin{tabularx}{\linewidth}{| p{0.22\textwidth} | X | X | p{0.25\textwidth} | X | l |}
		\hline
		Component & Failure Modes & Effects of Failure & Causes of Failure & Recommended Actions & SR \\
		\hline
		Authentication & Impersonated Superadmin manipulates user's database & User data is changed on back-end, or deleted & Database Security Failure & Reset superadmin password and rollback database & AR2 \\
		\hline
		Recommendation System & Recommendation misses preferences & Group will be given a recommendation which does not match all preferences & Preference Error & Group has to try a new recommendation or modify their preferences as none would match & ALGR1 \\
		\hdashline
		~ & Recommendation generation takes too long & Group is given recommendations too slowly & Algorithmic Efficiency Error & Server must be able to handle influx of requests at busy times & ER1 \\
		\hline
		~ & Recommendation generation is incorrect & Group or individual is given recommendations that do not meet their preferences at all & Algorithmic Error & Review and recreate error to determine where the algorithm is making mistakes and fix & ALGR2 \\
		\hline
		\end{tabularx}
\end{table}
\label{table3}
\newpage
\begin{table}[hp]
	\caption{Failure Modes and Effects 3} \label{TblFMEA3}
	\begin{tabularx}{\linewidth}{| p{0.17\textwidth} | X | X | p{0.25\textwidth} | X | l |}
		\hline
		Component & Failure Modes & Effects of Failure & Causes of Failure & Recommended Actions & SR \\
		\hline
		Physical Device & Application Crashes & Unsaved user data can be lost & General browser crash & Reopen browser application and fill in any data that was not saved & IR6 \\
		\hline
		Deployments & Pipeline Not Automatically Run & The current build of will look like it has no issues but the tests were not run & GitHub Error & Manually start pipeline & IR4, IR5 \\
		\hline
		\end{tabularx}
\end{table}
\end{landscape}

\section{Safety Requirements}
Below are the Requirements that have been formed by the above analysis.

\subsection{Access Requirements}
These requirements ensure that user data is only accessible to the correct users (superadmin and the user them self).
\begin{itemize}
	\item AR1: Users can only access and modify their own data.
	\item AR2: Only a superadmin can modify the database directly, which there is only one of.
\end{itemize}

\subsection{Integrity Requirements}
These requirements revolve around the user's data, our database, our deployment, and the user's device. These requirements ensure that the application maintains its own health, the health of the user's device, and the health of the data.
\begin{itemize}
	\item IR1: User data is not modified without their permission. \hyperref[table1]{(In table)}
	\item IR2: Database backups occur daily. \hyperref[table1]{(In table)}
	\item IR3: Database backups are kept for at minimum one month. \hyperref[table1]{(In table)}
	\item IR4: CI/CD Pipeline is run before every deployment to ensure a healthy application state. \hyperref[table3]{(In table)}
	\item IR5: CI/CD Pipeline is run on every new code change before it can be merged. \hyperref[table3]{(In table)}
	\item IR6: Application crashes will not cause the device to stop working. \hyperref[table3]{(In table)}
	\item IR7: Database will be available as long as the service is available. \hyperref[table1]{(In table)}
\end{itemize}

\subsection{Privacy Requirements}
This requirement ensures that access to the application data and user data requires proper authentication.
\begin{itemize}
	\item PR1: Users have to login with their credentials to access application data. \hyperref[table1]{(In table)}
\end{itemize}

\subsection{Efficiency Requirements}
This requirement ensures that users do not have to wait a long period of time before receiving their recommendation, preventing the user from thinking that the page has frozen.
\begin{itemize}
	\item ER1: Algorithm must complete the generation of recommendations in a reasonable time (Less than 5 seconds) and be able to report that information to the UI. \hyperref[table2]{(In table)}
\end{itemize}

\subsection{Algorithm Requirements}
These requirements ensure that the algorithm functions in edge cases and will ensure proper functionality of the algorithm.
\begin{itemize}
	\item ALGR1: Algorithm must be able to generate recommendations even if there is no perfect match to the user or group's preferences. \hyperref[table2]{(In table)}
	\item ALGR2: Algorithm must generate recommendations that align in some way with the user or group's preferences. \hyperref[table2]{(In table)}
\end{itemize}


\section{Roadmap}
The safety requirements determined within this document will be considered throughout the development of the project. After completion of key components (Frontend, Backend, Database etc.), hazard analysis will be conducted to ensure that potential risks are mitigated. If any issues or risks are discovered, action will be taken immediately to resolve them.


\end{document}